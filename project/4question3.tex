\section{Question 3: Markov possibility}
\textit{We try to model the	same system with a Markov process. Can we take into account all the events described in the table below?}	
 \begin{center}
\line(1,0){250}
\end{center}
One of the big advantages of using the markov state diagram approach is its versatility. Virtually every system can be modelled with it. The size and complexity can increase into a level in which calculation by hand is not feasable. However, it is then still a good method to illustrate a system and its possible states and transitions, without using it for further calculations. An exemplary markov state diagram of such purpose can be found in figure \ref{fig:markov_complex}.

\begin{figure}[!ht]
\centering
\begin{tikzpicture}[->, >=stealth', auto, semithick, node distance=4cm]
\tikzstyle{every state}=[fill=green!40,draw=black,thick,text=black,scale=1]
\tikzstyle{fail}=[circle,fill=red!40,draw=black,thick,text=black,scale=1]
\node[state]    (1)                     {$SSSLVV$};
\node[fail]     (L)[left of=1]          {$SSS\bar{L}VV$};
\node[fail]    (CC)[below of=1]        {$\bar{S}\bar{S}\bar{S}LVV$};
\node[state]    (S)[right of=1]         {$SS\bar{S}LVV$};
\node[fail]    (SS)[below of=S]        {$S\bar{S}\bar{S}LVV$};
\node[state]    (SV)[right of=S]       {$SS\bar{S}L\bar{V}V$};
\node[fail]    (SSV)[below of=SV]      {$S\bar{S}\bar{S}L\bar{V}V$};
\node[state]    (V)[above of=SV]        {$SSSL\bar{V}V$};
\node[fail]    (LV)[above of=V]        {$SSS\bar{L}\bar{V}V$};
\node[fail]    (VCC)[left of=LV]       {$\bar{S}\bar{S}\bar{S}L\bar{V}V$};
\node[fail]    (SVV)[right of=SV]      {$SS\bar{S}L\bar{V}\bar{V}$};
\node[fail]    (SLV)[above of=SVV]     {$SS\bar{S}\bar{L}\bar{V}V$};
\path
(1)     edge[bend left, below]    node{$\lambda_{LS}$}               (L)
        edge[]              node{$\lambda_{PSHCC}$}          (CC)
        edge[]              node[yshift=0.35cm]{$3(\lambda_{PSH})$}         (S)
        edge[bend left]     node{$2(\lambda_{SV}+\lambda_{SDV})$}(V)
        node[above=0.3cm]{1}
%
(S)     edge[]              node{$2(\lambda_{PSH})$}        (SS)
        edge[]              node[yshift=0.35cm]{$2(\lambda_{SV}+\lambda_{SDV})$}(SV)
        node[above=0.3cm]{4}
%        
(SV)    edge[]              node{$2(\lambda_{PSH})$}        (SSV)
        edge[]              node[yshift=2.5cm][xshift=-1.5cm]{$\lambda_{PSHCC}$}      (VCC)
        edge[below]         node[yshift=-0.35cm]{$(\lambda_{SV}+\lambda_{SDV})$}(SVV)
        edge[]              node[yshift=1cm][xshift=1cm]{$\lambda_{LS}$}(SLV)
        node[above=0.3cm]{6}
%
(V)     edge[]              node[yshift=0.6cm]{$3(\lambda_{PSH})$}    (SV)
        edge[]              node[yshift=1.6cm][xshift=0.9cm]{$\lambda_{PSHCC}$}    (VCC)
        edge[]              node{$\lambda_{LS}$}(LV)
        node[above=0.3cm]{9}
%
(L)     edge[bend left, above]    node{$\mu_{LS}$}               (1)
        node[above=0.3cm]{2}
%
(LV)    edge[bend left]    node{$\mu_{LS}$}               (V)
        node[above=0.3cm]{11}
%
(CC)    node[above=0.3cm]{3}
(SS)    node[above=0.3cm]{5}
(SSV)   node[above=0.3cm]{7}
(SVV)   node[above=0.3cm]{8}
(VCC)   node[above=0.3cm]{12}
%
(SLV)   edge[bend left]    node{$\mu_{LS}$}               (SV)
        node[above=0.3cm]{10};
%\node[above=0.3cm] (1){\small{1}};
%\node[yshift=-3.5cm, above of=1] (L){\small{1}};
%\node[yshift=-3.5cm, above of=L] (L){\small{2}};
%\node[above=0.5cm] (L){\small{2}};
%    edge[]             node{$\lambda_{PSHCC}$}      (E)
%(B) edge[]             node{$2\lambda_{PSH}$}           (C)
%    edge[]             node{$2(\lambda_V+\lambda_{DV})$}   (D)
%(C) edge               node{$1$}           (D)
%(D) %edge[loop right]    node{$(1-q)^2$}     (D)
    %edge[bend right,right]    node{$q(1-q)$}      (B)
    %edge[bend left]     node{$q(1-q)$}      (C)
%    edge[bend left,above]     node{$\mu_L$}         (A);
%\draw[red] ($(D)+(-1.5,0)$) ellipse (2cm and 3.5cm)node[yshift=3cm]{Patch H};
%;
\end{tikzpicture}
\caption{Complex markov state diagram }\label{fig:markov_complex}
\end{figure}

Each state is given in the notation $SSSLVV$. The three $S$ indicate the condition of the pressure sensors, the $L$ gives information about the functionality of the logic solver and the $VV$ part represents the two valve sets. Working states are indicated by green color and failed states by red color.

This expanded diagram allows to illustrate the various changes of partial failed states into other partial failed states and subsequently fully failed states.
As an example, the system can be in a functioning state with one pressure sensor failed and transit into another functioning state with one valve set failed.
This diagram also includes the possible repair of the logic solver.

\newpage

% Please add the following required packages to your document preamble:
% \usepackage{multirow}



%$\begin{bmatrix}
%  \scriptscriptstyle{-(\lambda_L+\lambda_{PSHCC}+3\lambda_{PSH}+2(\lambda_{SV}+\lambda_{SDV}))} & \lambda_L & \lambda_{PSHCC} & 3\lambda_{PSH} & 0 & 0 & 0 & 0 & \scriptscriptstyle{2(\lambda_{SV}+\lambda_{SDV})} & 0 & 0 & 0 \\
%  \mu_L & -\mu_L & 0 & 0 & 0 &0 & 0 & 0 & 0 & 0 & 0& 0\\
%    0 & 0 & 0 & 0 & 0 &0 & 0 & 0 & 0 & 0 & 0& 0 \\
%    0 & 0 & 0 & 0 & 0 &0 & 0 & 0 & 0 & 0 & 0& 0 \\
%    1 & 2 & 3 & \scriptscriptstyle{-(2\lambda_{PSH}+2(\lambda_{SV}+\lambda_{SDV})) & 2\lambda_{PSH}} &\scriptscriptstyle{2(\lambda_{SV}+\lambda_{SDV})} & 7 & 8 & 9 & 10 & 11& 12 \\
%    1 & 2 & 3 & 4 & 5 &6 & 7 & 8 & 9 & 10 & 11& 12 \\
%    1 & 2 & 3 & 4 & 5 &6 & 7 & 8 & 9 & 10 & 11& 12 \\
%    1 & 2 & 3 & 4 & 5 &6 & 7 & 8 & 9 & 10 & 11& 12 \\
%    1 & 2 & 3 & 4 & 5 &6 & 7 & 8 & 9 & 10 & 11& 12 \\
%    1 & 2 & 3 & 4 & 5 &6 & 7 & 8 & 9 & 10 & 11& 12 \\
%    1 & 2 & 3 & 4 & 5 &6 & 7 & 8 & 9 & 10 & 11& 12 \\
%    1 & 2 & 3 & 4 & 5 &6 & 7 & 8 & 9 & 10 & 11& 12 \\
% \end{bmatrix}$
 

%  $\begin{bmatrix}
% 1 & 2 & 3 & 4 & 5 &6 & 7 & 8 & 9 & 10 & 11& 12 \\
%  1 & 2 & 3 & 4 & 5 &6 & 7 & 8 & 9 & 10 & 11& 12
% \end{bmatrix}$