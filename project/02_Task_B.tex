\section{Null Hypothesis}
\textit{Explain how one may use the logrank test to test the null hypothesis that patients with the two tumour types have the same survival distributions.\\
Use the version of the test considered in Slides 6, p. 44-50, and do the necessary calculations for lifetimes up to (and including) 15.\\
It can be shown (you need not do it) that the total number of expected deaths for type 1 is 22.48 and for type 2 is 19.52.\\
Use this to calculate the test statistic of the logrank test, and perform the test with significance level 5\%.\\
What is the lowest significance level that will lead to rejection of the null hypothesis, i.e., what is the p-value?}
\begin{center}
\line(1,0){250}
\end{center}
la bla here

