\section{Question 5: Reliability}
\textit{Calculate the reliability of the system at the steady state.}
\begin{center}
\line(1,0){250}
\end{center}
The matrix formula
$\begin{bmatrix}
P1 & P2 & P3 & P4
\end{bmatrix}$
$\times$
$\begin{bmatrix}
 -(\lambda_S+\lambda_V+\lambda_L) & \lambda_S & \lambda_V & \lambda_L\\
 0 & 0 & 0 & 0\\
 0 & 0 & 0 & 0\\
 0 & 0 & 0 & 0\\
 \end{bmatrix}$
 $= 0$
 
results in following 5 equations, including the boundary condition V:


\begin{table}[!ht]
\centering
\label{tab:equations}
\begin{tabular}{ll}
I   & $-P1\times(\lambda_S+\lambda_V+\lambda_L) = 0$ \\
II  & $P1\times\lambda_S = 0$ \\
III & $P1\times\lambda_V = 0$ \\
IV  & $P1\times\lambda_L = 0$ \\
V   & $P1+P2+P3+P4 = 1$ 
\end{tabular}
\end{table}

From above equations the steady state reliability is $0$. This is independent of the actual numerical values for the various failure rates. This conclusion is trivial, as from the logical approach this result can be achieved easier: A system which only includes failure rates, but no repair rates, will eventually reach (and be kept in) a failed state if time steps into infinity.